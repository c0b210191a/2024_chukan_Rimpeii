% twocolumn を使うと2段組になる

%\documentclass[a4j,twocolumn]{jsarticle}        % -> platex
%\documentclass[a4j,twocolumn]{ujarticle}       % -> uplatex
\documentclass[uplatex]{jsarticle}   % -> uplatex + jsarticle

\usepackage{resume} % 他パッケージ,専用コマンド,余白の設定が書かれている

%%%%%%%%%%%%%%%%%%%%%%%%%%%%%%%%%%%%%%%%%%%%%%%%%%%%%%%%%%%%%%%%%%%%%%%%
% ヘッダ: イベント名,日付,所属,タイトル,氏名
%%%%%%%%%%%%%%%%%%%%%%%%%%%%%%%%%%%%%%%%%%%%%%%%%%%%%%%%%%%%%%%%%%%%%%%%

\pagestyle{plain}
\newcommand{\comment}[1]{}
\begin{document}
\twocolumn[
\beginheader{令和6年度 コンピュータサイエンス学部 中間発表}{2024}{8}{8}{井上 研究室}
\title{水温・振動提示によるフィードバック}
\author{C0B21019 井上 琳平 (Rimpei Inoue) }
\endheader
]

\vspace{3mm}

 % 本番用ページ番号オフセット
\setcounter{page}{1}

%---------------------------------------------------------------------------
% 本文
%---------------------------------------------------------------------------


\section{はじめに}
ゲームや読書に熱中すると,いつの間にか数時間経っていることがある.
この現象は物事に深く没入していると生じ,集中力の増加やストレス発散に繋がる.
仮想空間内での没入体験も同様のメリットが存在する.

VRの没入感向上には,接触感覚が重要である.VR体験中に

海外の研究では
没入感が低かったものの中でLight Rain,Heavy Rain,Slime Dripping on Backに着目すると,どれも液体であり,冷たい


空気で没入感があがった冷。
再現度低いもので液体らしいものがあること






VR体験での没入感向上の一環として,温度感覚に着目した研究がある.
従来の温度を提示する手法ではペルチェ素子を用いられることが多い.
ペルチェ素子は直流電流を流すと片面が加熱し,もう片面が冷却する.
直流電流の向きを変えると加熱と冷却が切り替わる.
直流電流を流すという簡単なプロセスで反応するため,手軽に使えることが利点である.
しかし,ペルチェ素子は加熱と冷却の切り替えがスムーズに行われない.

この問題を解決する温度提示手法として水がある.
あらかじめ,温水と冷水が用意されており,電子弁で流水を制御できるので,高速で温度を切り替えることが出来る.
また,接触感覚を伴わない温度の提示や,全身への温冷感の提示を可能としている.



\section{関連研究}
Yi-Ya Liaoらの研究では,水中にいる感覚を再現するLiquidMaskを開発した.
LiquidMaskは,HMDとユーザーの間に置かれる触覚インターフェースである.
ウォーターバッグに溜まった水が温度提示を行う.
更に,LiquidMaskが微細に振動することで,大型の魚が布巾を通過した際の揺れを再現している.
これによって,ユーザーはVR体験中の没入感が向上する.

Alexandraの研究では,様々な接触感覚を再現するForce Jacketを開発した.
Force Jacketはには26個のエアバッグが装着されており,それぞれが独立して稼働する.
エアバッグに空気が溜まり,ユーザーに圧力が提示される.
更に、エアバッグに附随している振動モーターが振動を提示する.
これによって,様々なイベントの再現が可能となり,VR体験の没入感が向上する.


\section{システム概要}
\begin{figure}[htbp]
    \centering
    \includegraphics[width=0.9\linewidth]{system.png}
    \caption{システム概要}
    \label{fig:env}
\end{figure}
システム概要を図\ref{fig:env}に示す.冷水と温水が入ったタンクを1つずつ用意する.水はポンプで吸い上げられ,サーモスタット混合栓を通る.この時,冷水と温水の量が調整されることで,幅広い水温を実現する.混合した水はチューブを通り循環し,温度刺激を与える.今回の研究では,衣服にチューブを取り付けることで,接触感覚無しで温度情報を提示する.


\section{システム構成}
\begin{figure}[htbp]
    \centering
    \includegraphics[width=0.9\linewidth]{システム構成.png}
    \caption{システム構成}
    \label{fig:sys}
\end{figure}
図\ref{fig:sys}はシステム構成である.仮想空間内の温度情報をコンピュータが保持している.まず,保持している温度情報がArduinoに伝えられる.そして,Arduinoがサーモスタット混合栓を動かし,温度の調整をする.最後に,温度を調整された水が温度刺激を行う.


\section{評価方法}
\begin{table}[H]
    \centering
    \caption{アンケート項目}\label{tab:fuga}
    \begin{tabular}{c|c}\hline
    番号 & 項目  \\ \hline
    1 & 濡れた感覚があった  \\
    2 & 痛みを感じた  \\
    3 & 接触感覚があった  \\
    4 & 温度の切り替わりが早かった  \\
    5 & 全身に温冷感が生じた  \\
    6 & どちらの手法で没入感がより向上した(春)  \\
    7 & 以下同文(夏)  \\
    8 & 以下同文(秋)  \\
    9 & 以下同文(冬)  \\ \hline
    \end{tabular}
\end{table}

Unityの仮想空間で実験を行う.実験参加者は,春夏秋冬を再現した4つの空間で温度情報を提示される.この時,2通りの温度提示手法を使う.1つ目は,従来の温水,または冷水を用いた温度提示である.2つ目は私が提案した,それぞれの気温に合わせて水温を調整し,提示する手法である.実験終了後,アンケートを行う.

\tabref{tab:fuga}はアンケートによる評価項目の一覧である.1番目は水を流した際の不快感の有無を確認している.2番目は温度提示の際にありがちな,痛覚が生じるかを確認している.3,4,5番目はどれも水を使った温度提示のメリットが再現されているかを確認している.6から9番目は仮想空間の温度提示において,上述したどちらの温度提示手法がより没入感を得られるかを確認している.


\section{検討事項}
適切な仮想空間が作れるか

\section{まとめ}
水を使った温度提示手法は,接触感がなく,全身に温冷感を伝えることが出来,高速で温度を切り替えれることを可能とした.この特性を消すことなく,更に多くの温度情報を伝えられる手法として,サーモスタット混合栓を使ったシステムを提案した.


%---------------------------------------------------------------------------
% 本文終わり
%---------------------------------------------------------------------------

 % 参考文献
\bibliographystyle{junsrt}
\bibliography{ref}


\end{document}


%-----------------------------------------------------
% テンプレート
%------------------------------------------------------------------------------

%-----------
%% 箇条書き
%-----------
%\begin{itemize}
% \item
%\end{itemize}

%-------------------
%% 番号付き箇条書き
%-------------------
%\begin{enumerate}
% \item
%\end{enumerate}

%-----------
%% 図の表示
%-----------
%\begin{figure}[H]
% \centering
% \includegraphics[clip,width=7cm]{hoge.eps}
% \caption{図タイトル}\label{fig:hoge}
%\end{figure}

%-----------
%% 図の参照
%-----------
%\figref{fig:hoge}

%-----------
%% 表の作成
%-----------
%\begin{table}[H]
% \centering
% \caption{表タイトル}\label{tab:fuga}
% \begin{tabular}{|c|c|c|}\hline
%  hemo & piyo & fuga \\ \hline
%  hemo & piyo & fuga \\ \hline
% \end{tabular}
%\end{table}

%-----------
%% 表の参照
%-----------
%\tabref{tab:fuga}

%-----------
%% 参考文献
%-----------
%\begin{thebibliography}{9}
% \bibitem{piyo} 参考文献
%\end{thebibliography}

%-----------------
%% 参考文献の参照
%-----------------
%\cite{piyo}
